\documentclass[
	%a4paper, % Use A4 paper size
	letterpaper, % Use US letter paper size
]{jdf}


\author{Shashvat Sinha}
\email{shashvat.sinha@gatech.edu}
\title{Assignment M2 (Summer 2020)\\CS6750}

\begin{document}
%\lsstyle

\maketitle

\begin{abstract}
    For my \textbf{M*} assignments in CS6750 (Summer 2020), I have chosen to redesign the interface that \textbf{LinkedIn} uses for its messaging. There are several aspects of the messaging interface that, in my opinion, could be improved - there is limited support for threading, messages do not support rich text, and the editor is geared towards short texts rather than a proper messaging platform. 
    
    As a networking tool LinkedIn would benefit from improved communication between its users, as it would improve the value proposition its users see. By the end of the M* assignments, I plan to focus on the task of communicating using LinkedIn messaging, taking it  through a complete design lifecycle. 
\end{abstract}

\section{Needfinding Execution}
We will perform the following NeedFinding executions:

\begin{itemize}
    \item Analysis of product reviews
    \item Participant observation
    \item Evaluation of existing user interfaces
\end{itemize}

Based on what we learnt in the needfinding via product reviews and participant observation,  exercised the alternative option described in \textbf{\emph{M1}}, the evaluation of existing user interfaces.


\subsection{Execution A: Analysis of Product Reviews}
\subsubsection{Overview}
A generic search on LinkedIn and LinkedIn messaging returned results of articles which were predominantly geared to how people were using LinkedIn, e.g. recruiters and marketers spamming users inboxes, rather than critiques of LinkedIn itself.

Further analysis and curation yielded the following reporting and reviews of the LinkedIn messaging redesigns in 2015 and 2017 that moved from email style messaging towards shorter chat style messaging.

\begin{table}[h] % [h] forces the table to be output where it is defined in the code (it suppresses floating)
	\caption{Analysis of Product Reviews}
	\small % Reduce font size
	\centering % Centre the table
	\begin{tabular}{L{0.25\linewidth} C{0.05\linewidth} L{0.15\linewidth} L{0.13\linewidth}  L{0.4\linewidth}}
		\textbf{Theme} & \textbf{Score} & \textbf{Citation} & \textbf{Review Date} & \textbf{Remarks}\\
		\toprule[0.5pt]
		Messaging interface redesign to favor short messages & 2 & \cite{hull_2015} & Sep 2015 & Reviewer goes through LinkedIn's (then) new messaging interface and gives it poor marks for execution.\\
		\midrule
		Walkthrough of LI's short messages & 4 & \cite{simos_2018} & Jul 2018 & Reviewer has positive view on LinkedIn short messaging\\
		\midrule
		Contact and messaging tags & 1 & \cite{driskell_2017} & Apr 2017 & User is unhappy with removal of message and contact tagging facility, describes how to leave LinkedIn\\
		\midrule
		LinkedIn's Facebook style messaging & 4 & \cite{gonzalez_2017} & Jun 2017 & Reviewer has positive view on LinkedIn short messaging, comparing it to Facebook's\\
		\midrule
		LinkedIn's messaging is now FB/Hangout style & 3 & \cite{ingraham_2020} & Feb 2020 & Reviewer describes how LinkedIn is now more like Facebook and Hangouts\\
		\midrule
		LinkedIn's messaging is now on the main page & 3 & \cite{lunden_2017} & Jan 2017 & Reviewer describes how chat is now on the main page to keep the user's attention on another feature, the news feed\\
		\midrule
		Walkthrough of LI's short messages & 4 & \cite{price_2017} & Apr 2017 & Messaging is designed to be everywhere on the LI interface instead of in its own tab\\
		\midrule
		Messaging on every page & 3 & \cite{warren_2017} & Jan 2017 & Messaging is on every page instead of on its own, similar to Facebook\\
		\midrule
		
	\end{tabular}
\end{table}

\subsubsection{Analysis}

The general themes are:
\begin{itemize}
    \item LinkedIn has moved from email style messaging to Facebook/Hangouts style short messages. 
    \item This move brings LinkedIn in line with its social networking peers.
    \item A second redesign also moves LinkedIn's messaging from its own page to all over the LinkedIn website, so that users can carry out their chat anywhere. This is a copy of Facebook's chat functionality.
\end{itemize}


\begin{enumerate}
    \item What kind of user has written the opinion or review.
    All but three reviews seem are written by staff writers for major online publications. In two cases the writers wrote on LinkedIn's own blog website. One case was of a user who wrote on their own website - this user was the most dissatisfied with the redesign.
    \item From what lens does the writer view LinkedIn - do they consider it a networking tool, a recruitment tool, or something else? 
    The reviews are written from a point of view where LinkedIn is considered a social networking tool, where its peer is Facebook.
    \item What is the writer attempting to do with LinkedIn? Is it related to the messaging capability?
    As staff writers, the reviewers (except one) are merely reporting the release of new features, rather that actually try to use LinkedIn as a service. There was only one exception to this (\cite{driskell_2017}), who found the redesign so detrimental to his productivity, he described how to migrate away from LinkedIn.
    
    \item What is the writer failing to do? Is there a Gulf of Execution and Gulf of Evaluation we can identify?
    The one writer who did have an open critique (\cite{driskell_2017}) had a clear Gulf of Execution, which could not be bridged after the redesign.
    
    \item What, if any, suggestions does the writer make as a means to make them whole again?
    The only writer with a critique (\cite{driskell_2017}) would have wanted LinkedIn's messaging to regain its tagging feature.
\end{enumerate}

\subsubsection{Conclusion}
The needfinding approach of \textbf{analysis of product reviews} did not return useful data.
There were very few actual product reviews that I was able to find of LinkedIn. The vast majority were critiques of LinkedIn's user culture and behavior, which does not help our exercise. The others were PR pieces written by staff writers who seemed to all say the same few things (e.g. comparisons with Facebook, the availability of messaging on all LinkedIn pages)

I would consider this needfinding approach for this topic to be sub-optimal.

\subsection{Execution B: Participant Observation}
I am a user of LinkedIn. Over the years, I have made several observations about LinkedIn's messaging system as I tried to use it for tasks related to networking and communication.

As a first step, I will collate the thoughts I have had over the years.

I will then take the outputs of the previous step - the Analysis of Product Reviews - and try to visit those experiences on myself. To do this, I will attempt to recreate the steps followed by the reviewers, or attempt to achieve the same end goal as them, and determine the areas that caused friction. This should give me a better understanding of the issues raised in those reviews.

I will then try to identify how my observations line up with the observations made by the other reviewers - and vice versa.

Here, special care must be taken to address confirmation bias - I already have a view on LinkedIn messaging (after all, I chose this topic for a reason). Therefore I must keep my eyes and mind open to possibilities that LinkedIn Messaging does meet my needs.

There is a second bias that I could be susceptible to - observer bias - but with a twist. Here, I would not be biasing users, or reviewers. After all, those reviews were written before this assignment was started and I had no interaction with the reviewers. Instead, it would be the reviewers biasing me, wherein having read those reviews, I would be predisposed to find the same faults they found.

My approach to mitigate the effects of this bias would be to split my Participant Observation exercise into two parts. The first part, where I put down my observations would be done \textit{before} the analysis of product reviews, so as not to bias myself.

The second part, where I try to visit the experiences of reviewers, and try them out myself would be done \textit{after} I have completed the analysis of product reviews.



\subsection{Execution C: Evaluation of Existing User Interfaces}
I'm not entirely sure of the efficacy of customer surveys for the objective of improving LinkedIn. 

Therefore as an alternative to user surveys, I would like to take propose a different needfinding exercise, that of evaluating other user interfaces.

There are several Personal Relationship Management (PRM) alternatives to LinkedIn, as well as examples from the CRM world. One could take a look at them to form a considered opinion.

To do this I would take the following steps:
\begin{enumerate}
    \item Perform a search for products that are trying to fill the gap in LinkedIn (my first exercise might give me starting points for the search).
    \item Gain access to these products (e.g. if they have free tiers or evaluation signups).
    \item Attempt to use these products in ways similar to how my survey participants have indicated a preference for, and also to address the findings of the Participant Observation and Product Reviews
    \item Note the Gulfs of Execution and Evaluation for these products.
\end{enumerate}

The reason this step can be important is because competing products are cases where someone felt strongly enough about LinkedIn's shortcomings that they put in the effort to build a better solution. This would be a stronger vote for a particular feature than merely an answer to a survey question or a review in a blog post. 

One can also assume that the building of a competing product to LinkedIn might have started with an exercise such as this one, and the end result that we see from the competitor is simply the distillation of that exercise into something actionable - which is what we are looking for in our own exercise.

If we are trying to improve our mousetrap, it may help to see how others are attempting to make a better mousetrap.




\section{Data Inventory}

\section{Defining Requirements}

\section{Continued Needfinding}

\section{References}

\printbibliography[heading=none]


\end{document}